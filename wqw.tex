\documentclass[11pt,a4paper]{moderncv}

% moderncv themes
%\moderncvtheme[blue]{casual}                 % optional argument are 'blue' (default), 'orange', 'red', 'green', 'grey' and 'roman' (for roman fonts, instead of sans serif fonts)
\moderncvtheme[blue]{classic}                % idem
\usepackage{xunicode, xltxtra}
\XeTeXlinebreaklocale "zh"
\widowpenalty=10000

% character encoding
%\usepackage[utf8]{inputenc}                   % replace by the encoding you are using
\usepackage{CJKutf8}
  
% adjust the page margins
\usepackage[scale=0.84]{geometry}
\recomputelengths                             % required when changes are made to page layout lengths
\setmainfont[Mapping=tex-text]{Noto Sans CJK SC}
\setsansfont[Mapping=tex-text]{Noto Sans CJK SC}
\CJKtilde

% personal data
\firstname{王全伟}
\familyname{}
\title{}               % optional, remove the line if not wanted

\mobile{(+86)188-4418-9533}                    % optional, remove the line if not wanted
\email{18844189533@qq.com}                         % optional, remove the line if not wanted
\homepage{www.github.com/jieyaren}                         % optional, remove / comment the line if not wanted
%% \quote{\small{``Do what you fear, and the death of fear is certain.''\\-- Anthony Robbins}}


\nopagenumbers{}

\begin{document}

\maketitle

\section{个人信息}
\cventry{学历}{本科2012--2016}{吉林大学电子信息工程系}{}{}{}  
\cventry{期望职位}{ C++后台工程师 }{}{}{面向分布式后台,分布式数据库,高并发后台开发的项目最好}{}
\cventry{}{期望工作地点: 深圳北京 }{}{}{期望薪资:  税前15k+}{}
%我在此项目负责了哪些工作,分别在哪些地方做得出色/和别人不一样/成长快,这个项目中,我最困难的问题是什么,我采取了什么措施,最后结果如何。这个项目中,我最自豪的技术细节是什么,为什么,实施前和实施后的数据对比如何,同事和领导对此的反应如何。
\section{工作与项目经历}
\cventry{}{海能达科技股份有限公司}{系统软件设计开发部}{}{}{}
\cventry{ 职责
}{1. 负责业务功能开发,测试与部署,代码静态检查{\newline}
2. 负责设计文档的编写{\newline}
3. 部门技术接口,与其他部门合作技术负责人{\newline}
4. 外场线上环境问题定位联系人,快速响应解决问题}{}{}{}{}
\cventry{2018至今}{一体化基站/微站开发,负责将成熟软件移植到PowerPC上,我在该项目中负责字节序转换与业务验证的工作,由于涉及到的消息结构体较多,我通过脚本收集整合相关结构体,生成对应的转序函数,快速完成任务}{}{}{}{}
\cventry{2016-2017}{从事PDT专网协议栈开发任务,我在该项目中负责核心网与基站后端应用开发,能高效的实现需求并自测验证,定位外场问题准确迅速}{}{}{}{}

\section{个人项目}
\renewcommand{\baselinestretch}{1.2}

\cventry{YAC}
{C++模板容器库}{}{}{}
{是一个c++容器类的实现,参考了boost::container https://github.com/jieyaren/yac}

\vspace*{0.2\baselineskip}

\cventry{Yet Another Algorithm}
{C++模板算法库}{}{}{}
{是一个C++算法库的一个简单实现,参考了boost::Algorithm和<algorithm> https://github.com/jieyaren/yaa}

\vspace*{0.2\baselineskip}

\cventry{toolkit}
{C++多线程基本组件}{}{}{}
{是一个POSIX api的简单封装,并实现了一些基本组件,如自旋锁等,参考了muduo https://github.com/jieyaren/toolkit}

\renewcommand{\baselinestretch}{1.0}

\section{技能}
\cventry{编程}{C/C++ $>$ Python $>$ Shell}{熟悉C++11,熟悉STL, boost, libevent, googletest 等}{}{}{}
\cventry{Linux}{熟练的Linux环境开发,熟悉GDB等调试工具}{}{}{}{}
\cventry{网络编程}{熟悉TCP/IP,多线程编程模型}{}{}{}{}
\cventry{数据库}{熟悉MySQL和Redis,了解原理实现}{}{}{}{}
\cventry{英语}{CET4, 查阅英语文献,github,stackoverflow无压力}{}{}{}{}

\closesection{}                   % needed to renewcommands
\renewcommand{\listitemsymbol}{-} % change the symbol for lists
\end{document}